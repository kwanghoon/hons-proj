% outline discussion
% 1. Symbolic restructuring
% 2. DQN results 
% 3. Problems with StarCraft
% 4. Can we make the interface better? 

\chapter{Discussion}

\section{Framework Evaluation}

The initial idea was to create modules to cluster part of the useful game state
in such a way that researchers would have only needed to ``configure'' a reward
function (if needed). Moreover one of the first design decisions turned out to
be slightly detrimental when we switched to the multi-client architecture,
making the implementation unnecessarily tricky and bug-prone.

\subsection{How can we improve this?}

The easiest but most painful solution (implementation-wise) to the design
problem would be to partially re-design the way the client handles the game,
while leaving the interface unchanged.

This could be achieved in different ways:

\begin{itemize}
\item The client could be split in three different asynchronous modules: a
  StarCraft handler to control Windows and the StarCraft process, a Game State
  collector to observe and collect the state using BWAPI and WinAPI, and a
  networking interface to deal with the external interface.
\item BWAPIClient could be expanded to fully control the game in a similar
  fashion.
\end{itemize}

\section{Challenges for machine learning approaches}

\subsection{State exploration as a policy}

\subsection{Hierarchical Decision-Making}

In Chapter 2 we have discussed some work in Hierarchical Reinforcement Learning
that could be used in principle to tackle StarCraft, however no particular model
seems to fit all the requirements of the platform, especially if we take and
end-to-end approach to the problem. It's probably fair to speculate that
similarly to AlphaGo, a reinforcement learning framework will need some amount
of strong supervision to learn good overall strategies and hand-tuned feature
extractors.

% Cover Russell work

\section{Future Work}

\subsection{Full Linux compatibility}

The great majority of AI and machine learning researchers do not use Windows as
their development environment. Having to use a VM to host the StarCraft client
and the interface means that there is a higher cost then normal associated with
using the platform (i.e. it's not straightforward to provide a simple
installation script or a working out-the-box environment).

In the past some parts of BWAPI were compatible with the WINE, a Linux
environment that tries to reproduce the Windows API and provide tools for
running Windows software on unix platforms. Porting the entirety of our pipeline
over Linux will mean being able to significantly simplify the integration with
other state-of-the-art machine learning systems. 

\subsection{ROS integration}

% picoros

% ros 2.0


\subsection{Policy Reuse}

% include also policy distillation?