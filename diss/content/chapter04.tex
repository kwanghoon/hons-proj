\chapter{Experimental Methodology} 

\section{Computing Resources}

The entirety of the development was done on a 8-core machine...

The Q-learning algorithm was tested on this same machine, while 

\section{Q-learning}

\subsection{Task dependent state representation}

Providing a state representation for quickly testing the whole system ended up
resulting in a far more challenging problem than initially estimated. To be able
to extensively test the symbolic game state data being sent by the client we
wanted to obtain a useful state representation without having to use a black-box
function approximator such as an artificial neural network. Using a neural
network would have made it very hard to do automatic testings. Moreover we
wanted to check and explore how hard it would have actually been to use standard
state-compression techniques for StarCraft's large state space.

\subsection{Multi-unit Q-learning}

\section{DQN}

\subsection{Multi-unit DQN}